\chapter{Introduction}
\hrule
\vspace{40pt}

\section{Centralized Markets \& The Orderbook}
Financial markets exist in many forms. Perhaps the most popular is the \textit{centralized} market, where all trading
activity is handled by a central \textit{trusted} authority. The advantage of using a centralized market is that prices
are kept fair and government/financial authority regulation is much easier to implement. With the advent of \textit{blockchain technology}\footnote{A blockchain is a distributed ledger/datastructure that stores transactions/contracts in a cryptographically secure and distributed way without the need for trust of a centralized organization.}, \cite{BTC2008},
\textit{Decentralized markets} have risen in popularity. Decentralized markets allow clients to cut out the middle man and directly trade with each other.
\textit{Automated market making} algorithms, \cite{ANGERIS2020}, are used to ensure the decentralized markets have sufficient liquidity for efficient operation.
Decentralized exchanges have the advantage of removing any need for trust of a third party, which is an advantage for markets such as the \textit{cryptocurrency}\footnote{Cryptocurrencies are digital currencies which are transacted using blockchain technology.} market,
where scams are common and exchanges have a tendency to collapse, \cite{FTX2024}. Of course the downside of decentralized exchanges is that they are harder to
regulate, \cite{SALAMI2021}, and therefore are less attractive to risk averse institutional investment. In this paper we concern ourselves with centralized exchanges
in the context of the cryptocurrency market.

Centralized exchanges use a \textit{limit orderbook} to organize all buying and selling activity. 
The limit orderbook is a datastructure that stores all unfilled buy and sell orders.
These orders sit on the orderbook at different levels and are indexed by price. This gives us a snapshot of the current supply and demand. It is well established that for most markets,
supply and demand are key drivers of price movement, \cite{MANKIW2014}. It then follows that we should be able to predict the future
price movement of a security by using only the information contained in its orderbook. This is the main aim of this paper.

\section{Related Works}
Since the introduction of electronic exchanges, orderbook modelling has been a widely studied active area of research.
Early works used parametric models such as \cite{CAO2009} who use autoregressive linear models with handcrafted features such as spread and trade imbalance
to forecast future mid-price returns at the minute resolution or \cite{HAWKES2013} who introduce Hawkes processes for modelling orderbook events. \cite{DAT2017} use
tensor regressions for high frequency orderbook mid-price returns forecasting. Due to their parametric nature, these models all inherently make
assumptions on the causal relationships that drive mid-price movement. With the explosion in popularity/practicality
of deep learning over the past few decades, \cite{RUMELHART1986}, \cite{LECUN1989}, \cite{HOCHREITER1997}, \cite{LECUN1998},  \cite{ALEX2012}, \cite{VASWANI2017},
we have seen several deep learning approaches to the orderbook modelling task. Deep learning has the benefit of having fewer assumptions/restrictions
on model architecture and excels at modelling large complex systems with high dimensional and non-linear dynamics, such as an orderbook.
\cite{DIXON2017} was one of the first works to use deep neural networks for mid-price prediction (specifically they used MLPs, \cite{ROSENBLATT1958}, \cite{HORNIK1989}, with several hidden layers).
\cite{PASSALIS2020} use a neural Bag-of-Features approach, achieving state of the art performance for high frequency mid-price returns prediction at the time.
A breakthrough in the field was made when \cite{ZHANG2019} proposed a specialized deep learning model, coined DeepLOB, for high frequency mid-price returns prediction.
The DeepLOB model uses CNNs, \cite{LECUN1998}, for feature extraction and LSTMs, \cite{HOCHREITER1997}, for temporal dependency extraction.
They showed that the DeepLOB model significantly outperforms all other state of the art models of the time. The DeepLOB model
uses the orderbook data in raw form. It turns out that this is not the best representation for predictive modelling. 
\textit{Orderflow}, is a stationary orderbook representation, first introduced in \cite{CONT2013} in the context of linear price impact modelling.
\cite{KOLM2023} build on the work of \cite{ZHANG2019} and adapt the DeepLOB model to use orderflow input, instead of the raw orderbook representation.
This adapted model is named DeepOF and \cite{KOLM2023} show that it significantly outperforms the DeepLOB model. \cite{LUCCHESE2024} further
build on this work by again modifying the DeepOF model to include extra volume features at deeper levels of the model.

The majority of these papers are trained and evaluated on the LOBSTER, \cite{LOBSTER2011}, NASDAQ orderbook dataset. This is a dataset
that has become the gold standard for orderbook modelling, since it is well researched, documented and easily accessible. 
In this paper we turn away from traditional finance and look to model cryptocurrency orderbook price movement.
The cryptocurrency market is still a very new area of research with far fewer published papers.
Of note are \cite{AKYILDIRIM2023} who use traditional machine learning models and \cite{SHIN2021} who use LSTM based models
to forecast Bitcoin mid-price movement. It is worth noting that these papers focus on forecasting movement for longer time periods,
such as minutes, hours or even days. To our knowledge, there are no published papers on \textbf{high frequency} mid-price movement prediction, akin to
\cite{ZHANG2019}, \cite{KOLM2023}, \cite{LUCCHESE2024}, but for \textbf{cryptocurrency data}. 

\section{Overview}
In this paper we explore various machine learning models for predicting high frequency returns
for four of the main cryptocurrency futures trading pairs, BTCUSDT, ETHUSDT, MATICUSDT and SOLUSDT. 
We construct a novel data set from a live WebSocket stream from Binance, the worlds largest centralized cryptocurrency exchange.
Following \cite{CONT2013}, we define a stationary transformation of the raw orderbook data called orderflow, 
and motivate its definition by showing that under a stylized model of the orderbook, orderflow can be used to measure linear price impact
of orderbook events. We then define our modelling objective as a three-class classification problem, predicting the direction of the smoothed
mid-price returns for several prediction horizons. For this task, we introduce several traditional machine learning and also deep learning
models that use either the orderflow representation or the raw orderbook representation. Specifically, from traditional ML we introduce
Logistic Regression and XGBoost models. For our deep learning models, we use the deepLOB model from \cite{ZHANG2019} and the deepOF model
from \cite{KOLM2023}, which use the raw orderbook/orderflow representations respectively.
We train, validate and test our models for each trading pair and multiple prediction horizons on eight windows from 05-02-2024 to 25-02-2024.
We present the results on the test set and also evaluate the feature importances for each XGBoost model.

\section{Contributions}
In this paper we present two main contributions.
\begin{enumerate}
    \item The first major contribution we make is the application of high frequency limit orderbook modelling to cryptocurrency data.
        Specifically we train and evaluate models for classifying the direction of the smoothed mid-price change for prediction horizons ranging from
        $\approx 500$ milliseconds to $\approx 20$ seconds for Binance perpetual futures limit orderbook WebSocket data.
    In the literature all similar modelling, \cite{ZHANG2019}, \cite{KOLM2023}, \cite{LUCCHESE2024}, has been done on traditional equities data, such as the NASDAQ, \cite{LOBSTER2011}.
    To our knowledge, we are the first paper to explore high frequency orderbook forecasting on cryptocurrency data.
    Our dataset is also unique, since it has been gathered from a live WebSocket, \cite{WEBSOCKET2011}, endpoint and to our knowledge,
    there have been no similar papers published on forecasting live orderbook data.
    This is exciting, since this live data is openly available and anyone can create a connection and start receiving
    live updates. This massively lowers the barrier to entry for practical trading applications and is one of the benefits
    of trading on cryptocurrency exchanges. Also, often times a historical dataset may be used for model evaluation but then,
    in practice, no live feed is available with sufficient latency and or resolution. Since we are directly using data
    stored from a live feed, we can guarantee that such a data source will be available for downstream trading strategies.
    \item The second contribution we make is the use of XGBoost, \cite{XGBOOST2016}, for high frequency orderbook modelling. 
    Within the quantitative finance industry, XGBoost is a very popular model and is very widely used. 
    It has the ability to model high dimensional complex non-linear feature relationships with tighter control of overfitting
    and greater explainability when compared to deep learning methods. For many quantitative finance applications model explainability
    is extremely important. It is a necessity to be able to understand and explain when and why a model is not performing as expected,
    especially when one has to answer to angry investors.
    In this paper we show that XGBoost with orderflow features can rival state of the art deep learning models, \cite{ZHANG2019}, \cite{KOLM2023}, \cite{LUCCHESE2024}, for this application.
\end{enumerate}



